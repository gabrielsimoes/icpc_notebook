\section{Notes}
\subsection{Modular Multiplicative Inverse}
\begin{itemize}
\item If~$gcd(a, m) = 1$, then let~$ax + my = gcd(a, m) = 1$ (Bezout's Theorem). Then~$ax \equiv 1{\pmod {m}}$.
\item If~$gcd(a, m) = 1$, then~$a \cdot a^{\phi (m)-1}\equiv 1{\pmod {m}}$ (Euler's Theorem).
\item If~$m$ is prime, then ~$\phi (m) = m - 1$, so~$a * a^{m - 2} \equiv 1 {\pmod {m}}$.
\end{itemize}

\subsection{Chinese Remainder Theorem}
We are given~$N = n_1 n_2 \cdots n_k$ where~$n_i$ are pairwise coprime. We are also given~$x_1 \cdots x_k$ such that~$x \equiv x_i {\pmod {n_i}}$.
Let~$N_i = N/n_i$. There exists~$M_i$ and~$m_i$ such that~$M_i N_i + m_i n_i = 1$ (Bezout). Then, there is only one solution~$x$, given by:
$x=\sum\limits_{i=1}^{k}a_{i}M_{i}N_{i}$

\subsection{Euler's Totient Function}
Positive integers up to a given integer~$n$ that are relatively prime to~$n$.
$\varphi (n)=n\prod\limits_{p\mid n}\left(1-{\frac {1}{p}}\right)$ where the product is over the distinct prime numbers dividing~$n$.

\subsection{Möebius}
If~$F(n) = \sum\limits_{d | n}{f(d)}$, then
$f(n) = \sum\limits_{d | n}{\mu(d) F(n / d)}.$

\subsection{Burnside}
Let~$A \colon GX \rightarrow X$ be an action. Define:
\begin{itemize}
\item $w \coloneqq $ number of orbits in~$X$.
\item $S_x \coloneqq \{g \in G \mid g \cdot x = x \}$
\item $F_g \coloneqq \{x \in X \mid g \cdot x = x \} $
\end{itemize}
Then $ w = \frac{1}{|G|} \sum\limits_{x \in X}{|S_x|} = \frac{1}{|G|} \sum\limits_{g \in G}{|F_g|}. $

\subsection{Catalan Number}
$C_n$ is solution for:
\begin{itemize}
	\item Number of correct bracket sequence consisting of $n$ opening and $n$ closing brackets.
	\item The number of rooted full binary trees with $n+1$ leaves (vertices are not numbered). A rooted binary tree is full if every vertex has either two children or no children.
	\item The number of ways to completely parenthesize $n+1$ factors.
	\item The number of triangulations of a convex polygon with $n+2$ sides (i.e. the number of partitions of polygon into disjoint triangles by using the diagonals).
	\item The number of ways to connect the $2n$ points on a circle to form $n$ disjoint chords.
	\item The number of non-isomorphic full binary trees with $n$ internal nodes (i.e. nodes having at least one son).
	\item The number of monotonic lattice paths from point $(0,0)$ to point $(n,n)$ in a square lattice of size $n×n$, which do not pass above the main diagonal (i.e. connecting $(0,0)$ to $(n,n)$).
	\item Number of permutations of length $n$ that can be stack sorted (i.e. it can be shown that the rearrangement is stack sorted if and only if there is no such index $i<j<k$, such that $a_k<a_i<a_j$ ).
	\item The number of non-crossing partitions of a set of $n$ elements.
	\item The number of ways to cover the ladder $1\dots n$ using $n$ rectangles (The ladder consists of $n$ columns, where $i^{th}$ column has a height $i$).
\end{itemize}
Recursive: \[C_0=C_1=1\] \[C_n= \sum_{k=0}^{n-1}C_kC_{n-1-k}, n\geq2\]
Analytical: \[C_n=\frac{1}{n+1}{2n \choose n}\]

\subsection{Landau}
There is a tournament with outdegrees ~$d_1 \leq d_2 \leq \ldots \leq d_n$ iff:
\begin{itemize}
\item $d_1 + d_2 + \ldots + d_n = {n \choose 2}$
\item $d_1 + d_2 + \ldots + d_k \geq {k \choose 2} \quad \forall 1 \leq k \leq n.$
\end{itemize}
In order to build it, let~1 point to~$2, 3, \ldots, d_1 + 1$ and repeat recursively.

\subsection{Erdös-Gallai}
There is a simple graph with degrees~$d_1 \geq d_2 \geq \ldots \geq d_n$ iff:
\begin{itemize}
\item $d_1 + d_2 + \ldots + d_n$ is even
\item $\sum\limits_{i = 1}^k{d_i} \leq k(k-1) + \sum\limits_{i=k+1}^n{\min(d_i, k)} \quad \forall 1 \leq k \leq n$.
\end{itemize}
In order to build it, connect~1 with ~$2, 3, \ldots, d_1 + 1$ and repeat recursively.

\subsection{Gambler's Ruin}
In a game in which we win a coin with probability~$p$ and lose a coin with probability~$q \coloneqq 1 - p$, the game stops when we win~$B$ ou lose~$A$ coins. Then~$\mathit{Prob}(\textnormal{win B}) = \frac{1 - (p/q)^B}{1 - (p/q)^{A+B}}$.

\subsection{Extra}
\newcommand{\Fib}{\mathit{Fib}}
\begin{itemize}
\item $\Fib(x + y) = \Fib(x + 1) \Fib(y) + \Fib(x) \Fib(y - 1)$
\end{itemize}
